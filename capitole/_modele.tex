Baza de date a fost gândită în jurul a trei entități: cererea, decizia și vânzarea.

Cererea este scrisă de clientul ce are produsul avariat și îl dorește înlocuit sau reparat.
În cerere trebuiesc completate următoarele date:
\begin{itemize}
	\item Date personale.
	\item Descrierea avariei și a evenimentului ce a produs avaria.
	\item Existența sau nu a altei polițe de asigurare.
\end{itemize}
Este legată relațional de imagini încărcate pentru:
\begin{itemize}
	\item Paginile contractului.
	\item Raportul poliției în caz de furt agravat.
	\item Copia bonului cu ștampilă.
	\item Copia actului de identitate.
	\item Pozele produsului.
\end{itemize}
Aceasta poate să fie ori respinsă, ori acceptată, construindu-se astfel o decizie.

Decizia este construită în momentul acceptării cererii de despăgubire.
Conține datele relevante pentru rapoartele de vânzare, precum:
\begin{itemize}
	\item Prețul reparației.
	\item Fostele decizii asociate. (i.e: foste reparații)
	\item Franciza.
	\item Data livrării produsului în service.
	\item Costul inspecției și a curierului.
\end{itemize}
Aceasta, pentru o bună perioadă de timp, era legată obligatoriu de o decizie și de o vânzare.

De la un punct, când trebuia să se importe toate vechile vânzări din raportul păstrat în Excel, s-a luat decizia de a se decupla aceste două entități.
Se construiau până în acel moment vânzări generice, ce nu existau.
Practic, existau într-un raport vânzările respective, dar nu puteau fi găsite din cauza lipsei informațiilor din partea rapoartelor de la Altex / Media Galaxy.
Nu se putea lega IMEI-ul sau numărul facturii, din cauza modului de organizare a organizației.
Aveau nevoie urgentă de rapoartele de vânzare, așa că am fost anevoios de acord să decuplez legătura dintre decizie și vânzare.

Mai nou decizia conține numele produsului, prețul, data achiziționării și perioada asigurată, deși informațiile tot se citesc din tabelul de vânzări.

Vânzarea conține:
\begin{itemize}
	\item Numele produsului.
	\item Prețul produsului.
	\item Data achiziționării.
	\item Perioada de asigurare.
	\item Numele clientului (ce poate lipsi)
\end{itemize}
O vânzare poate conține una sau mai multe asigurări, precum unul sau mai multe produse.



