\section{Decuplarea deciziilor de vânzări}

	La scurt timp după ce a intrat modulul de legare a deciziilor de vânzări în funcțiune, structura bazei de date s-a izbit de prima problemă a lumii reale, datele incomplete sau lipsă.

	Utilizatorii sistemului nu găseau vânzările asociate cererilor, din cauza incompatibilităților datelor săptămânale trimise de Altex / Media Galaxy.
	Problema aceasta a fost accentuată de adăugarea obligativității de a asocia decizia cu cererea de despăgubire.
	Înainte de a putea modifica orice câmp ce ținea de cerere, mulțumită dezvoltării treptate, trebuia astfel găsită asocierea, pentru a reduce din numărul cererilor fără decizie, pentru a recupera din datoria tehnică.

	De la un punct, când trebuia să se importe toate vechile vânzări din raportul păstrat în Excel, s-a luat decizia de a se decupla decizia de vânzare.

	Am încercat pentru un scurt moment de a modifica interfața de asociere a cererii și a vânzării pentru a forma o despăgubire, adăugând opțiunea de creere a unor vânzări generice.
	Problema apare când se dorește scoaterea unui raport de vânzare într-o anumită perioadă, deoarece nu se știe magazinul ce a vândut produsul.

	Deci, într-un final, am decuplat decizia de vânzare.
	Am renunțat la structura strâns legată a cererilor de vânzări, în favoarea căutării manuale.
	Fiecare decizie acum conține detalii și despre produs și despre asigurare.

	Altă soluție mai elegantă nu s-ar fi putut găsi, deoarece lipsa de informații ce n-ar putea să difere din cadrul cererii de despăgubire și a datelor primite nu există.

	\subsection{Modificări în viitorul apropiat}

	Doresc să reintroduc în viitorul apropiat descoperirea automată a deciziilor vechi înapoi prin adăugarea unui câmp de factură la asocierea unei noi cereri de o vânzare.

	Diferența acum, față de sistemul decuplat, este că pot să aflu ce decizii au fost legate înainte, mulțumită serviciului de backup, dar și a modului actual de legare a informațiilor, pentru că informațiile legate de vânzare nu au fost modificate până acum de mână.

	Avantajul existenței opționale a câmpului de factură ajută la funcționarea algoritmului elegant de găsire a deciziilor, pentru că ar fi același număr de factură, adică aceeași vânzare.
	Normal că ar trebui să fie trecut prin prisma utilizatorului sistemului, deoarece s-ar putea referi factura la un alt produs și o altă asigurare.
	De aceea sistemul ar fi opțional, precum sistemul actual de IMEI duplicat.

