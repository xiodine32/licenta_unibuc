\section{După planul inițial de dezvoltare}

	În urma discuțiilor și finalizării modulelor inițiale, s-a finalizat dezvoltarea modulelor pe orizontală, dar s-a mutat pe verticală.
	A dezvolta pe orizontală reprezintă dezvoltarea modulelor și a legăturii dintre ele, iar a dezvolta pe verticală reprezintă aprofundarea unui domeniu (în cazul acesta, a construirii rapoartelor și a introducerii datelor).

	După o întâlnire, s-a decis introducerea fostelor facturi plătite în sistem și ulterior posibile planuri pentru un sistem modular de rapoarte.


	\subsection{Introducerea datelor și rapoartele avansate}

	La fel ca în cazul datelor primite de la Altex / Media Galaxy, rapoartele primite (de data asta complete) au determinat o parcurgere completă a lor și modificarea coloanelor ce existau într-un raport, dar lipseau dintr-altul.

	Mulțumită numărului redus de rapoarte dorite s-a reușit parcurgerea lor și asigurarea unui algoritm mai simplu din punctul de vedere al condițiilor posibile.

	S-au mai descoperit probleme în momentul introducerii lor de două ori in baza de date -- operație ce n-ar fi trebuit să mai arate diferențe între datele introduse și cele deja existente.
	Unele câmpuri se modificau de două ori.
	De vină a fost greșeala umană și după ce s-a discutat cu persoanele responsabile de rapoarte s-au mutat pe server-ul de producție.

	\subsection{În continuare}

		Chiar dacă s-a terminat dezvoltarea și pe verticală, mai există o mică problemă nu-mi dă pace: problema tabelelor prea mari.
		Tabela cu cele mai multe informații, a cererii, are în momentul actual 48 coloane.

		Doresc să construiesc o migrare cu următoarele tabele:
		\begin{itemize}
			\item Persoană
			\item Eveniment
			\item Factură
			\item Produs
			\item Garanție veche
		\end{itemize}
		Aceste tabele vor avea o relație unu-la-unu cu tabela cererii.
		Se vor încărca din baza de date doar valorile necesare.
		Se va putea implementa mult mai ușor și statistici pentru fiecare persoană, eveniment sau factură.

		Legat de statistici, se dorește după o scurtă pauză de dezvoltare implementarea unui sistem modular de rapoarte.
		S-ar defini prin intermediul unei interfețe grafice ce câmpuri să fie scoase, din ce tabele, cu ce legături între ele.
		Rapoartele vor fi salvate în Excel și se vor putea reutiliza șabloane pentru cele mai des folosite.

		Practic, se dorește înlocuirea sistemului actual de rapoarte cu unul nou, modular.
