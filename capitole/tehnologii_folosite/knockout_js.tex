\section{Knockout.js}

	Knockout este o implementare de sine stătătoare a structurii șablon de „Model-View-ViewModel”.
	Structura se referă la o separare clară a datelor primite din exterior, a componentele și a datelor ce vor fi folosite pentru a fi afișate, dar și la prezența unui cod specializat de a gestiona relațiile dintre cele două (componente și interfață). \cite{knockout_js}

	A fost concepută de „Steve Sanderson”, un angajat Microsoft, drept un proiect cu codul distribuit gratuit. \cite{knockout_js_site}
	Mulțumită arhitecturii, Knockout simplifică relațiile complexe dintre componentele vizuale, ce ajută aplicația de a fi mult mai echilibrată și rapidă.

	Am folosit Knockout pentru a rezolva problema codului duplicat.
	În momentul actual, este folosit pentru a gestiona pozele încărcate de utilizator și de a salva extern baza de date.
	Am ales tocmai în această arie cu probleme majore, pentru că dacă o componentă nu ar fi mers, tot sistemul ar fi fost inutilizabil.
	Dar mulțumită codului modular și a folosirii șabloanelor ce repetă același element cu alte modele legate în spate, Knockout asigură funcționalitatea identică pentru toate modulele folosite.

	Sistemul de gestiune a pozelor este împărțit în cele patru categorii:
	\begin{itemize}
		\item Facturi.
		\item Pozele cu produsul avariat.
		\item Foile cu termenii și condiții.
		\item Raportul poliției, în caz de nevoie.
	\end{itemize}

		exportul backup-ului
