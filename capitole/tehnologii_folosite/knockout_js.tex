\section{Knockout.js}

	Knockout este o implementare de sine stătătoare a structurii șablon de „Model View ViewModel”.
	Structura se referă la o separare clară a datelor primite din exterior, a componentele și a datelor ce vor fi folosite pentru a fi afișate, dar și la prezența unui cod specializat de a gestiona relațiile dintre cele două (componente și interfață). \cite{knockout_js}

	A fost conceput de „Steve Sanderson”, un angajat Microsoft, drept un proiect cu codul distribuit gratuit. \cite{knockout_js_site}
	Mulțumită arhitecturii, Knockout simplifică relațiile complexe dintre componentele vizuale, ce ajută aplicația a fi mai echilibrată și rapidă.

	Am folosit Knockout pentru a rezolva problema codului duplicat.
	În momentul actual, este folosit pentru a gestiona pozele încărcate de client și de a salva extern baza de date.
	Am ales să folosesc librăria această zonă unde pot apărea probleme, pentru a asigura performanța și eliminarea erorilor ce pot apărea în timpul procesului de programare.
	Codul este astfel modular și folosește șabloane pentru a lega componentele vizuale cu structuri de date.
	Knockout asigură astfel funcționalitatea identică pentru toate modulele folosite.

	Sistemul de gestiune a pozelor este împărțit în cele patru categorii:
	\begin{enumerate}
		\item Facturi.
		\item Pozele cu produsul avariat.
		\item Foile cu termenii și condiții.
		\item Raportul poliției, în caz de nevoie.
	\end{enumerate}
	Pentru fiecare element, se construiește o funcție nouă, pentru că JavaScript nu are clase.
	Funcția respectivă gestionează identic modelul și vizualizarea acestuia prin păstrarea detașată de celelalte modele a datelor proprii.

	Pentru a putea încărca asincron, câte o poză pe rând, în mediul distribuit de stocare a datelor (cloud), am combinat Knockout.js cu jQuery.
	Astfel, datele sunt primite de la server, în format JSON, folosind o cale specială, determinată la momentul construirii rutelor.

	Knockout fiind o librărie de sine stătătoare, nu oferă o legătură strânsă între datele primite de la server și reprezentarea internă a structurii, dar permite extensibilitatea cu orice alt limbaj, pentru că se folosește doar de spațiul de nume („namespace”) \verb|ko|.
	Dar mulțumită librăriei ajutătoare jQuery și a metodelor scrise de mine, se asigură o performanță și un număr redus de probleme ce pot apărea, pentru orice sistem al oricărui client, ce poate vizita website-ul.

	Un alt exemplu unde s-a folosit Knockout este la salvarea baza de date din cadrul aplicației.
	Knockout primește de la server o listă cu toate tabelele bazei de date, care se modifică dinamic.
	Soluția de backup este accesibilă numai administratorului de sistem.
	Administratorul poate salva separat un anumit tabel sau integral baza de date.
