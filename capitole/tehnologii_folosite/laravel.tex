\section{Laravel 5.3}

	Laravel este o librărie gratuită de PHP.
	Pentru Laravel 5.3, minimul necesar librăriei este PHP 5.5.9 sau mai nou.
	Librăria a fost concepută de „Taylor Otwell”, pentru a construi aplicații web bazate pe arhitectura „Model-View-Controller”.
	Printre numeroasele avantaje de a folosi Laravel se află sistemul modular de pachete, cu propriul său sistem de gestionare, moduri diferite de a accesa bazele de date relaționale, utilitare pentru scrierea și menținerea aplicației și orientarea spre expresivitatea codului.\cite{laravel_book}

	Arhitectura „Model-View-Controller” se bazează pe cele trei componente inter-conectate slab ale unei aplicații: Modelul, Controller-ul, respectiv View-ul.
	Modelul este componenta centrală a șablonului, pentru că se ocupă de datele, logica și regulile aplicației.
	Acest model este gestionat de Controller, ce acceptă datele de la utilizator și le transformă în comenzi pentru model și/sau View.
	View-ul este orice metodă de vizualizare a informației unui Model.\cite{poo_sa}

	Avantajele arhitecturii, mulțumită separării conceptuale, ajută dezvoltarea simultană și ușurința modificărilor.
	Astfel, mai multe persoane pot lucra simultan la o parte a aplicației.
	Dar din păcate, trebuie studiată în detaliu, pentru că nu este ușoară de folosit pragmatic.

	\subsection{Directive Blade}
		Un mare avantaj al librăriei este un simplu, dar foarte puternic și eficient sistem de șablonare --- Blade.
		Spre deosebire de alte sisteme de șablon, Blade nu inhibă dezvoltatorul a folosi cod PHP în View-uri.
		De fapt, toate View-urile Blade sunt traduse în cod simplu PHP și salvate până ce sunt modificate.
		Astfel reușește să adauge aproape zero logică în plus aplicației.

		Pentru a afișa ceva pe ecran folosind PHP în mijlocul codului HTML, ar trebui să scrii:
			\begin{Verbatim}
			<?= $variabila ?>
			\end{Verbatim}
		Dar asta te face vulnerabil la „XSS”.

		XSS (Cross Side Scripting) este o vulnerabilitate a securității unei aplicații, des întâlnită în aplicațiile web, ce permite adăugarea oricărui cod, fie el malițios sau inocent, paginilor vizualizate de clienții normali.

		Astfel, pentru a preveni exploatarea datelor trimise de la Controller, pentru a afișa valorile variabilelor trimise, directivele Blade indică folosirea sintagmei:
			\begin{Verbatim}
			{{$variabila}}
			\end{Verbatim}

		Desigur, nu ești limitat la doar nume de variabile, ci la orice s-ar putea considera parametrul unei funcții, deoarece sunt trimise prin metoda \verb|htmlspecialchars| a PHP-ului.

		Printre cele mai folositoare beneficii folosind Blade se află și construirea fișierelor ce definesc structura HTML-ului aplicației.
		Nu se definește practic o pagină principală, doar se specifică pentru paginile existente deja a se folosi structura găsită în pagina specificată prin comanda:
		\begin{Verbatim}
			@extends('layouts.app')
		\end{Verbatim}
		unde în acest caz în folderul „layouts” găsim fișierul „app.blade.php”.

		Pentru a introduce conținutul paginii ce extinde șablonul, Blade folosește secțiuni.
		Declarate în pagina principală, acestea indică locul unde se așteaptă conținut, putând fi denumite după nevoile șablonului.
		De asemenea, în cazul definirii unei secțiuni, se poate folosi și o valoare implicită.
		Cel mai des e folosit un șablon în definirea titlului, cu o valore implicită a numelui proiectului.
		În cazul acestui proiect, am folosit ca sufix numele configurat în fișierul mediu, lăsând la dispoziția fiecărei pagini a seta titlul.

		Folosind secțiuni, am descoperit o problemă la refolosirea unor componente vizuale.
		Blade adaugă precum PHP conținutul fișierului ce trebuie să existe unde ai scris instrucțiunea respectivă folosind comanda:
		\begin{Verbatim}
@include('decisions.includes.search')
// ce adaugă fișierul
// resources/views/decisions/includes/search.blade.php
		\end{Verbatim}

		Continuând șirul dezvoltării, descopăr o altă problemă la introducerea a mai multor segmente de cod JavaScript în aplicație, din fișiere Blade separate.
		Pentru a rezolva și această problemă, Blade se folosește de:
		\begin{Verbatim}
			@push('scripts')
		\end{Verbatim}
		pentru a adăuga informații într-o stivă, ce va fi afișată în pagina ce definește structura HTML prin simpla comandă:
		\begin{Verbatim}
			@stack('scripts')
		\end{Verbatim}

		Printre marile îmbunătățiri aduse de Blade, folosite în acest proiect, se numără și directivele.
		Acestea sunt propriile comenzi, definite de programator, ce extind limbajul.
		Sunt traduse în simple instrucțiuni de cod PHP.
		Am scris astfel propria directivă de a construi automat codul necesar afișării, folosind HTML și librăria vizuală „Bootstrap”, a unui câmp dintr-un formular, cu nume, tip și validare proprie:
		\begin{verbatim}
			@input(['eveniment_strada', ['Strada', false]])
			// unde 'eveniment_strada' - id-ul câmpului.
			// unde 'Strada' - denumirea câmpului.
			// unde 'false' - necesitatea completării câmpului.
		\end{verbatim}
		ajutatoare pentru Bootstrap / panel-uri
	\subsection{Laravel Forms / HTML}

		O mare parte a oricărei aplicații web este interacțiunea cu clientul.
		Pentru asta, există formulare ce sunt trimise către calculatorul aplicației.
		O fostă librărie inclusă în Laravel, dar acum extinsă și desprinsă este librăria ajutătoare pentru formulare și HTML.
		Este întreținută de „Laravel Collective”, o comunitate ce se ocupă de menținerea componentelor scoase din Laravel de „Taylor Otwell”.
		Astfel, el are prilejul de a se concentra pe munca sa (Laravel) și comunitatea susține componentele îndrăgite de programatori în viață, independent.\cite{laravel_collective_about}

		Un astfel de proiect este „Forms \& Html” \cite{laravel_collective_forms}, ce simplifică codul ce trebuie scris de un programator Laravel.
		În loc de a scris de fiecare dată ce câmp are ca referință ce obiect din model, se definește în schimb în momentul construirii formularului modelul de referință.
		Când se construiește un câmp, trebuie doar să fie bine specificat identificatorul câmpului, iar când va fi trimis către client, va fi completat automat cu valoarea din model.
		De asemenea, un al treilea parametru lasă loc programatorului să completeze celelalte atribute ale nodului HTML printr-un vector unde cheia reprezintă numele atributului și valoarea reprezintă conținutul.

		Nu doar metode ajutătoare, „Forms \& Html” acceptă și componente specializate, definite de programator, ca șabloane Blade.
		Prin definirea lor într-un folder special, acestea pot fi folosite precum cele deja definite, pentru a ușura clasele și atributele necesare pentru dezvoltatorii ce folosesc „Bootstrap”.
		Cum acest proiect se folosește de Bootstrap și „Forms \& Html”, vă pot da un exemplu de două componente des folosite în aplicație:
		\begin{Verbatim}
{{Html::panel_begin('Delete Assurance', 'in')}}
  // aici scriu cod...
{{Html::panel_end()}}
// unde 'Delete Assurance' - numele panoului
// unde 'in' - starea panoului: de a fi deja deschis
		\end{Verbatim}

	\subsection{Artisan}
		migrations - artisan - database migrations
		artisan make:commands
		artisan routes:list
		artisan make:auth
	\subsection{Eloquent ORM}
		models - Eloquent ORM
