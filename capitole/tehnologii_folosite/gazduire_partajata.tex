\section{Găzduire partajată}

		Găzduirea partajată se folosește pentru a pune un web server conectat la internet valabil pentru mai multe persoane.
		Partajându-se mai multor clienți, se amortizează costul îngrijirii și astfel prețurile sunt reduse.

		Am ales pentru Fandu soluția oferită de xServers\cite{xservers} pentru certificatele ce asigură calitatea (ISO 9001) și securitatea (ISO 27001) informației, suport tehnic non-stop, infrastructură și echipamente profesionale, cu o disponibilitate de 99.9\% lunară, viteza aplicației fiind asigurată și de către soluția de stocare adoptată --- SSD („Solid State Drive”).

		Un mare avantaj, dar în același timp și dezavantaj este oprirea comenzii de interpretare și executare a codului PHP, \verb|eval|.
		Din păcate, este problematic pentru consola Artisan, pentru că nu poate să aplice în momentul actualizării bazei de date tot codul de migrare.

		Soluția, la care s-a ajuns drept compromis în urma dezbaterilor cu departamentul tehnic, ce migrează codul și baza de date a server-ului, este:
		\begin{enumerate}
			\item
				Se salvează toate datele existente, folosind utilitarul oferit de xServers.
			\item
				Se introduce un fișier nou în directorul aplicației la:
				\begin{verbatim}
					storage/framework/down
				\end{verbatim}
				Se închide astfel accesul la server cât timp se migrează baza de date și conținutul aplicației în sine.
			\item
				Se află codul necesar migrării bazei de date prin comanda:
				\begin{verbatim}
					php artisan migrate --pretend -vvv
				\end{verbatim}
				Ce reiese va fi cod SQL ce va fi copiat cu ușurință în phpMyAdmin-ul de pe server.
			\item
				După ce s-a migrat cu succes baza de date, se copiază următoarele foldere:
					\begin{itemize}
						\item \verb|app| - tot codul important al aplicației.
						\item \verb|bootstrap| - sistemul de încărcare a platformei Laravel.
						\item \verb|config| - setările implicite Laravel, în lipsa valorilor în fișierul mediului curent (\verb|.env|).
						\item \verb|database| - migrările bazei de date
						\item \verb|public| - pentru că folosim „Laravel Elixir”, ce se ocupă de gestiunea variantelor codului JavaScript și CSS, se vor copia și în calea publică \verb|public_html|, pentru că acestea vor fi servite clienților și trebuie să poată să fie accesibile.
						\item \verb|resources| - directivele Blade, codul JavaScript și SASS.
						\item \verb|routes| - rutele ce determină Controller-ul apelat.
						\item \verb|storage| - doar structura, pentru că va conține fișierele temporare traduse din Blade în PHP.
						\item \verb|tests| - se poate sări, pentru că nu afectează rezultatul final.
						\item \verb|vendor| - dacă s-a modificat \verb|composer.json|, adăugând sau modificând sau ștergând pachete.
					\end{itemize}
			\item
				Se va șterge fișierul nou creat:
				\begin{verbatim}
					storage/framework/down
				\end{verbatim}
				și astfel migrația va fi completă.
		\end{enumerate}

		În cazul unor migrări minore, ce nu implică baza de date, se pot suprascrie direct fișierele modificate în cadrul aplicației.
		PhpStorm ajută aici prin opțiunea de a încărca prin FTP (encriptat) fișierele modificate.

		\textbf{\underline{Observație:}} Se copiază directorul \verb|public| de două ori, când se fac modificări la stil sau la codul JavaScript: o dată în cadrul directorului ce conține întregul proiect Laravel și o dată în directorul \verb|public_html|.

		„Laravel Elixir” funcționează bazându-se pe un fișier JSON, ce spune ce versiune de stil și cod să folosească, aflat la:
		\begin{verbatim}
			/public/build/rev-manifest.json
		\end{verbatim}

		Această cale nu este valabilă clienților, pentru că aplicația nu se află în directorul public.
		S-au făcut deci mici modificări asupra fișierului \verb|index.php|, aflat tot în directorul public, pentru a arăta calea corectă.

		Trebuie să se copieze, în concluzie, codul și stilul și în acest director.
