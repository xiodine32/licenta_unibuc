\section{Baza de date - MySQL}



	Baza de date relațională cu sursa sa deschisă MySQL este un sistem de gestionare a unei baze de date relaționale.
	Numele provine de la numele fiicei fondatorului „Michael Widenius”\cite{mysql_reference} și abrevierii SQL („Structured Query Language”).
	Dezvoltarea sa, cu codul sub termenii licenței GNU GPL, a fost sponsorizată și proprietatea companiei suedeze „MySQL AB”.
	În momentul actual, compania nu mai există, fiind cumpărată de corporația Oracle.

	Această componentă vitală face parte din platforma cu codul deschis pentru dezvoltarea aplicațiilor web LAMP:
	\begin{itemize}
		\item Linux
		\item Apache
		\item MySQL
		\item PHP/Perl/Python
	\end{itemize}
	Multe aplicații, de la cele pentru amatorii paginilor personale, a buletinelor de știri sau a canalelor de discuție, se bazează pe MySQL.
	Printre marile companii ce folosesc MySQL, se numără:
	\begin{itemize}
		\item Google
		\item Facebook
		\item Twitter
		\item Flickr
		\item YouTube
	\end{itemize}


	\subsection{Relații}

	În Octombrie 2005, corporația Oracle a achiziționat „Innobase OY”, o companie finlandeză ce a dezvoltat InnoDB. \cite{innodb}
	Sistemul dezvoltat de salvare permite bazei de date MySQL a oferi tranzacții și „chei străine”.
	După achiziție, Oracle a confirmat prelungirea contractului ce permite folosirea lui de către „MySQL AB”.

	Cheile străine, în contextul bazelor de date relaționale, se referă la un câmp sau mai multe ce identifică unic o coloană sau o linie a unui tabel de alt tabel\cite{foreign_key}.
	De asemenea, un tabel poate să aibă una sau mai multe referințe, ce se pot referi la alte tabele părinte, dar chiar și la el însuși (numindu-se „cheie străină recursivă”).
	Pentru a păstra legăturile între tabele, atunci când se modifică date sau se șterg, se oferă următoarele opțiuni:
	\begin{itemize}
		\item \verb|CASCADE| - se șterge și referința.
		\item \verb|RESTRICT| - nu se poate șterge linia atâta timp cât există o referință spre el.
		\item \verb|NO ACTION| - asemănătoare \verb|RESTRICT|, ce se verifică în schimb la finalizarea tranzacției.
		\item \verb|SET NULL| - referința se va transforma în \verb|NULL|.
		\item \verb|SET DEFAULT| - referința se va transforma în variabila cu care este inițializat câmpul în mod implicit.
	\end{itemize}

	\subsection{PhpMyAdmin}

	PhpMyAdmin este un sistem gratuit și cu codul liber de administrare a bazelor de date MySQL și MariaDB.
	Ca o aplicație portabilă scrisă majoritar în PHP, a devenit una dintre cele mai populare sisteme de administrare, mai ales pentru serviciile de găzduire, precum xServers și nu numai.

	A început dezvoltarea „Tobias Ratschiller”, un consultant IT și mai târziu fondator a companiei Maguma, a unei interfețe bazate pe PHP a bazei de date MySQL în 1998.
	În anul 2000, din cauza lipsei de timp, a renunțat la proiect, dar până în acel moment, devenise deja una dintre cele mai populare aplicații administrative.
	Având o comunitate mare, un grup de trei dezvoltatori, pentru a coordona mai bine numărul care tot creștea de modificări, înregistrează la SourceForce „The phpMyAdmin Project” și preiau dezvoltarea în 2001. \cite{phpmyadmin}

	În continuare, principalul website a migrat spre un sistem distribuit și au mutat ulterior urmărirea problemelor pe GitHub.
	Înainte de versiunea 4, ce folosește apeluri asincrone pentru a optimiza folosința, se foloseau cadre HTML.

	În afară de a gestiona bazele de date și a oferi o interfață web, mai poate să:
	\begin{itemize}
		\item importe datele din CSV și SQL.
		\item să salveze într-o multitudine de formate, printre care și CSV, SQL, XML, Excel.
		\item caute orice date dintr-un context global sau local.
		\item transforme orice „blob” într-un format predefinit.
		\item ofere grafice actualizate în timp real pentru monitorizarea:
			\begin{itemize}
				\item conexiunilor
				\item memoriei
				\item consumului procesorului
			\end{itemize}
	\end{itemize}
