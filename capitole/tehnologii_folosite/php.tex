\section{PHP 5.6}
	La apariția internet-ului, toate paginile erau statice.
	Nu puteai să interacționezi cu conținutul paginii.
	Dar apoi Netscape a revoluționat paginile, construind limbajul de programare LiveScript (redenumit ulterior JavaScript).

	În puțini ani, Netscape a revoluționat și modul în care paginile sunt distribuite utilizatorilor prin conceptul de JavaScript pentru aplicațiile ce serveau paginile.
	Astfel, s-a construit primul scenariu de cod folosit de servere. \cite{TheInformationRevolution}

	În continuare, s-a definit „Common Gateway Interface” (CGI), ce oferă un protocol pentru serverele web de a executa programe pentru a genera pagini de web dinamice, actualizate în funcție de cererea utilizatorului.
	Serverul web permite administratorului să seteze ce URL-uri vor fi folosite de ce programe CGI prin setarea unui folder cu scripturi.
	În loc să trimită fișierul respectiv, serverul HTTP apelează script-ul și trimite orice ar fi afișat ca și rezultat pentru client. \cite{cgi}

	Un astfel de program a fost și invenția lui Rasmus Lerdorf.
	Acesta a extins programele ce gestionau pagina lui personală pentru a interacționa cu baze de date și formulare, dând numele implementării „Personal Home Page/Forms Interpreter” (sau pe scurt, PHP/FI).
	Însă acesta nu a avut o viziune clară asupra limbajului de programare, a continuat să adauge limbajului de programare ceea ce considera a fi următorul pas logic. \cite{php_code}

	Printre îmbunătățirile aduse limbajul se numără „superglobals” (o metodă ușoară de a accesa parametrii trimiși de formularele paginilor HTML) în versiunea 4.1, metode de gestiune a codului orientat pe obiecte și o interfață standardizată pentru a accesa bazele de date.
	Ulterior s-a adăugat pe parcursul dezvoltării limbajului suport pentru „namespace”-uri, generatoare, funcții anonime, extinderi.
	În 2014, a fost lansat PHP 5.6, ce va fi întreținut până pe 31 Decembrie 2018. \cite{php_supported_versions}
