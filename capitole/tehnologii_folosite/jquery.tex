\section{jQuery \& jQuery UI}

	Orice programator ce dorește o pagină interactivă o să folosească o dată măcar în viața lui cu jQuery.
	jQuery este o librărie de JavaScript ce ajută la: \cite{jquery}
	\begin{itemize}
		\item Navigarea DOM-ului („Document Object Model” --- toate componentele ce formează pagina)
		\item Găsirea ușoară a elementelor relative, precum a părinților sau a unor copii a unor noduri, folosind sintaxa similară CSS („Cascading Style Sheet”).
		\item Construirea ușoară a animațiilor și gestionarea ușoară a anunțurilor rezultate (și nu numai)
		\item Ușoara integrare a elementelor preluate dinamice din altă parte, folosind metodologia asincronă de a trimite și recepționa informații.
	\end{itemize}

	De asemenea, mulțumită arhitecturii, jQuery permite dezvoltatorilor a construi „plug-in”-uri, pentru a extinde funcționalitatea.
	O astfel de librărie ce extinde jQuery este „jQuery UI” \cite{jquery_ui}.

	jQuery UI este o colecție de elemente vizuale, animate, testate, ce permit dezvoltatorilor un timp mai scurt de a pune produsul pe piață.
	Oferă efecte profesionale și „widget”-uri (elemente vizuale predefinite), precum un modul de a putea selecta data dintr-un calendar vizual.

	Pentru sistemul de gestiune a cererilor de despăgubire, am folosit „jQuery” și „jQuery UI” pentru majoritatea interacțiunilor specifice paginilor.
	Am și extins platforma, prin trei metode, pentru a trimite și primi mai ușor informațiile de la Laravel.

	Spre exemplu, logica de a încărca vânzările săptămânale se bazează pe această platformă.
	Utilizatorul o dată ce adaugă fișierul dorit de a fi încărcat în sistem, nu navighează de la pagina curentă, ci este întâmpinat cu o bară a progresului.
	În acest moment, fiecare fișier este încărcat separat și preluat de sistem.
	Se salvează rezultatul pentru fiecare fișier și la final se combină toate rezultate într-o interfață ce ajută persoana ce folosește sistemul de a înțelege ce a funcționat și ce nu.

	Mai este folosit jQuery exhaustiv pentru a construi legătura dintre o vânzare și o cerere de despăgubire.
	Sistemul astfel refolosește codul deja existent prin intermediul cadrelor HTML („iframe”) pentru a selecta vânzarea.
	La final, reies mai multe câmpuri obligatorii, ce se completează automat dacă a fost ales o cerere de vânzare.
	Din cauza problemelor rapoartelor de vânzare, această abordare oferea cea mai mare flexibilitate și un număr minim de linii de cod scrise.

	O altă bună integrare cu „jQuery” vine din partea librăriei „Bootstrap Validator”, un „plug-in” simplu de folosit pentru a adăuga elemente de validare pentru formulare. \cite{boostrap_validator}
	Mulțumită directivelor Blade, am integrat ușor construirea unor câmpuri cu validare necesară, pentru a ajuta utilizatorul să completeze cât mai corect formularul de despăgubire.

	Am reușit astfel de a scurta timpul de dezvoltare și de a asigura o interfață unificată în modul de abordare a înfățișării aplicației, mulțumită acestor trei platforme.
