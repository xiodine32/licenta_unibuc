\section{Amazon Web Services}

	Inițial, tot proiectul ar fi fost într-o platformă distribuită.
	Dar problema datelor tot rămânea.
	Trebuia să asigur existența și valabilitatea lor pentru o perioadă de minim doi ani.
	Pentru asta, îmi trebuia un sistem distribuit de salvare a datelor.

	Inițial, mă gândeam să aleg soluția oferită de Azure, pentru că mai interacționasem cu ei.
	Dar după ce am mai prospectat piața, am ajuns la concluzia că ar fi mult mai benefic din punctul de vedere al costului întreținerii și a numărului mare de date să aleg „Amazon Web Services” --- S3.

	\subsubsection{Integrarea cu Laravel}

	Integrarea cu Laravel s-a făcut aproape instantaneu, mulțumită fișierului:
	\begin{verbatim}
	config/filesystems.php
	\end{verbatim}
	ce se ocupă cu gestionarea fișierelor aplicației.

	Tot ce a trebuit să se modifice a fost valoarea implicită pentru „cloud” și să se introducă în fișierul de mediu cheia (publică și secretă), precum și regiunea server-ului S3 oferit de AWS.

	Inițial gândisem un sistem de construire a unei imagini cu o rezoluție mult mai mică și a unei imagini normale ca dimensiuni, pentru a păstra cât mai mică dimensiunea imaginilor transferate.

	După o scurtă perioadă de dezvoltare, după ce s-a ajuns la o variantă finală a algoritmului, descoperisem că până la urmă, problema era constrângerea de \verb|5000| de acțiuni de a pune în sistemul de date.
	Optimizasem prematur o problemă inexistentă.
	Astfel am renunțat la construirea imaginii cu rezoluție mult mai mică și nu mai modific imaginea normală.
	Cu această modificare a crescut imens și numărul de tranzacții posibile de server-ul găzduit, pentru că nu mai trebuia să modifice fiecare imagine în parte.

	Dar eram limitați și de numărul de „GET”-uri (i.e: afișarea unei poze de către un client web).
	Am setat ca fiecare obiect încărcat să fie agresiv salvat în memoria clientului web.
	Atunci când se redeschide cererea de despăgubire, se află deja salvat în „cache” și nu numai că nu mai face o cerere sistemului de date, dar se și încarcă instantaneu, ca prin magie.

	O problemă ușor rezolvabilă este ștergerea datelor vechi.
	Am propus și încă aștept un răspuns din partea firmei ca atunci când o cerere de despăgubire nu va mai fi valabilă, să poată să fie salvate pe calculatorul unei persoane autorizate din firmă datele încărcate de client și șterse din sistemul distribuit.
