\section{Certificat SSL}
	Certificatele SSL sunt fișiere mici ce leagă digital o cheie criptografică de detaliile unei organizații.
	Atunci când este instalat pe un server web, activează protocolul HTTPS și permite legăturilor encriptate de la server la clienți.
	De obicei, SSL este folosit pentru a securiza tranzacțiile cu cărțile de credit și transferul de date sau de înregistrări.
	Dar mai nou, este normal să fie folosit la orice, de la media la paginile personale.\cite{ssl}

	Certificatele SSL leagă împreună:

	\begin{itemize}
		\item Un nume de domeniu sau server.
		\item O identitate organizațională (i.e: numele companiei) și locația sa.
	\end{itemize}

	O autoritate de certificate este o entitate ce distribuie certificate digitale SSL.
	Acesta acreditează deținătorul asupra unei chei publice cu numele său.
	Astfel se asigură prin semnături digitale anumite presupuneri despre cheia privată.
	Autoritatea de certificate are aici rolul de terță de încredere, pentru deținător și nu numai.

	O astfel de autoritate de certificate este StartSSL.
	StartSSL face partea din compania StartCom, fondată de „Eddu Nigg” în Israel.
	Aceasta oferă gratuit certificate SSL clasa 1, ce funcționează pentru un singur domeniu al unui website.

	Mai nou se promovează „Let's Encrypt”, o altă autoritate de certificate, ce oferă gratuit certificate HTTPS (SSL/TLS), pentru a opri terțele din a vedea traficul de la client la server.
	Dar din păcate, pentru sistemele găzduite la comun, în acest caz xServers, această opțiune nu este valabilă și nici preferată.
	Pentru că oricine poate să semneze un certificat, se pune problema legării acelui certificat la identitatea organizațională.

	În concluzie, am preferat să mergem cu StartCom.

	\subsection{StartCom}

		La începutul dezvoltării aplicației, o mare necesitate era o puternică securitate a paginilor.
		Pentru că mai folosisem înainte StartCom și aveam o oarecare încredere în ei, am ales pentru a activa rapid modulul HTTPS pe gazda de producție un certificat de la ei.

		StartCom a fost achiziționat în secret de „WoSign Limited” (Shenzen, China), o altă autoritate de certificate.
		Din cauza numeroaselor probleme cu ei, Mozilla, Apple și Google au anunțat retragerea încrederii în certificatele după 21 octombrie 2016 (Mozilla, Google), 30 septembrie 2016 (Apple).
		În continuare, Google Chrome 57 va avea încredere în certificatele website-urilor în topul Alexa 1M, iar Chrome 58 va avea încredere în cele în topul 500000. \cite{chrome_wosign}

		Din păcate, acest proiect de gestionare a cererilor de despăgubiri a fost prins la mijloc, pentru că aveam o oarecare încredere în această autoritate de certificate.
		Apăruse astfel problema că după o perioadă de timp nu o să mai funcționeze sistemul HTTPS.

	\subsection{CloudFlare}

		Din fericire, știam o soluție temporară, ce după multe dezbateri cu personalul de la Fandu, a intrat în funcțiune cât de curând posibil.

		CloudFlare este o companie în Statele Unite ale Americii.
		Oferă un sistem de livrare al conținutului și un sistem distribuit de nume de domeniu.
		Sistemul distribuit de nume de domeniu stă între serverul administratorului aplicației și aplicația mobilă a clientului.
		Astfel, nu numai că poate să prevină orice atac malițios, dar poate să și optimizeze paginile trimise.

		Un alt mare avantaj pentru CloudFlare, introdus recent, este un sistem de gestiune a certificatelor SSL.
		Astfel, am putut folosi certificatul ce nu mai putea fi folosit pentru a face o legătură „Full SSL”:
		\begin{itemize}
			\item
			Se asigură legătura client -- CloudFlare prin certificatul lor.
			Apare iconița ce indică transmisia encriptată a datelor către server (lacătul verde).
			\item
			Se asigură legătura CloudFlare -- server prin certificatul deja existent pe Fandu.
			Nu se trimite nicio informație nesecurizată prin Internet.
		\end{itemize}

		De asemenea, mai oferă și redirectarea automată a website-ului către domeniul securizat, dar deja se ocupă de asta Laravel.

		Principalul motiv pentru care CloudFlare este în momentul actual indispensabil pentru Fandu este protecția de atacurile DDoS.
		Un atac DDoS („Distributed Denial of Service”) poate să aducă în genunchi întreaga arhitectură.
		Cum CloudFlare oferă o opțiune de „sunt sub asediu”, ce poate fi activată cât mai repede, poate să blocheze orice om frustrat.

		Mulțumită juxtapoziției între server și client, se mai poate optimiza orice resursă trimisă.
		Se poate renunța la spațiile inutile din stilul sau codul interactiv JavaScript.
		Dar nu este nevoie de optimizarea la nivel de CloudFlare, deoarece se face la nivelul dezvoltării aplicației de Laravel.

		Mulțumită „Laravel Mix”, se optimizează, concatenează și minimizează tot codul și stilul afișat utilizatorului.
		Cu cât mai puține dependențe externe, cu atât mai bine pentru securitatea aplicației.
