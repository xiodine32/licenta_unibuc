\section{Utilitare}


	Am folosit utilitare pentru a scurta din timpul de dezvoltare.
	Dacă pentru Laravel există consola Artisan, pentru gestionarea proiectului am folosit PhpStorm, un mediu de dezvoltare integrat oferit de cei de la JetBrains.

	Composer împreună cu npm au fost folosite pentru interfețe cu formatul Excel și asigurarea completarea câmpurilor obligatorii de client.

	Iar pentru lucrarea de licență am folosit Sublime Text 3 și \LaTeX .

	\subsection{Google Analytics}

		Pentru a înțelege traficul venind, am ales să folosesc analizatorul gratuit oferit de Google.
		Astfel, aceasta oferă detalii despre interacțiunea paginilor web, de la cât timp petrece un client web pe o anumită pagină, la cele mai des folosite butoane sau cele mai des folosite drumuri (legături între mai multe pagini). \cite{google_analytics}
		De asemenea, poți vedea ce a declanșat o eroare, ceea ce pentru un dezvoltator ajută la identificarea și rezolvarea problemelor din momentul în care apar, nu după ce primești un apel furios de la unul dintre fondatorii companiei.

		Urmărind traficul pentru o săptămână, că punctul sensibil și cel mai des folosit era în momentul în care se căuta o cerere de despăgubire.
		De cele mai multe ori, se știa deja numărul cererii, trebuia doar să se acceseze panoul de căutare și să se introducă numărul în căsuță.
		Astfel am modificat panoul de navigare principal, regăsit pe fiecare pagină, prin introducerea unui câmp „ID Claim”, ce oferă  aceeași funcție regăsită în panoul de căutare a modulului cererilor de despăgubire.

		Urmărind traficul, s-a mai observat durata lungă petrecută de clienți în momentul introducerii datelor cererii de despăgubire.
		Laravel a fost conceput cu un sistem avansat de gestionare a sesiunii, pentru a preveni abuzurile, dar aceasta interfera cu durata lungă în care clienții își completau probabil informațiile și apărea des o eroare a expirării sesiunii.
		Astfel am rezolvat problema apelând constant, prin jQuery, un punct al aplicației cu singurul rol de a actualiza sesiunea Laravel (declarând sesiunea activă).
		Clientul va putea în final să completeze toate datele, sesiunea rămânând activă.

	\subsection{PhpStorm IDE}

		PhpStorm este un mediu de dezvoltare comercial, oferit de JetBrains.
		Suportă completarea avansată a codului PHP 5.6 (și mai nou 7.0), dar și analiza sintactică, prevenirea erorilor.
		Suportă oficial și sistemul relațional de bază de date MySQL, stilul CSS, JavaScript și are puternice metode de restructurare a codului PHP.
		Integrarea cu modul de depanare „XDebug” îl ajută să ajungă în fața multor simple editoare de text.

		Laravel a fost gândit cu cât mai puține referințe, cu cât mai mult cod ajutător pentru dezvoltator și o grămadă de resurse valabile pe diverse site-uri.
		Directivele Blade, introduse de platforma Laravel, sunt înțelese de analizatorul sintactic al mediului de dezvoltare PhpStorm.

		Pentru a profita din plin de cunoștințele sintactice acestui mediu de dezvoltare, un pachet des întâlnit în PhpStorm când se construiește o aplicație folosind platforma Laravel este: \\
		„Laravel PhpStorm IDE Helper” (\verb|barryvdh/laravel-ide-helper|)

		„Eloquent ORM” de cele mai multe ori PhpStorm nu poate înțelege de ce a fost apelată o metodă ce nu există - pentru că se va interpreta codul scris în metoda magică \verb|__callStatic|, ce va traduce numele metodei statice într-un text, iar apoi într-o clasă ce va construi o cerere SQL.
		Dar aici intervine „phpDoc”, sistemul preferat de adnotare a funcțiilor, metodelor și variabilelor PHP.
		Pachetul astfel introduce trei funcții ce pot fi apelate de dezvoltator prin intermediul consolei:
			\begin{itemize}
				\item
				\verb|php artisan ide-helper:generate| - generare cod „phpDoc” pentru fațade.
				\item
				\verb|php artisan ide-helper:models| - generare cod „phpDoc” pentru modele.
				\item
				\verb|php artisan ide-helper:meta| - generare cod „phpDoc” pentru structura aplicației.
			\end{itemize}

		De asemenea, se integrează cu sistemul de gestiune a versiunii codului.
		Poate să și încarce automat modificările făcute pe server-ul de producție, folosind protocolul FTP (encriptat, desigur).

		Asigură deci rapiditatea și fluența necesară reparării unor probleme critice și a dezvoltării codului concis, după ghidul dorit de stil.

	\subsection{Heroku}

		Pentru a minimiza impactul unei migrări, s-a folosit în cazul acestui proiect Heroku.

		Heroku este o platformă distribuită ce oferă platforma sa ca un serviciu („PaaS”).
		Această platformă oferă următoarele avantaje:

		\begin{itemize}
			\item Servicii pentru mai multe limbaje de programare, printre care și PHP.
			\item Implementarea ultimelor modificări preluate de pe GitHub, sistemul de gestiune a codului ales în cadrul proiectului.
			\item Acces în cadrul aplicației la consola Artisan, pentru a rula migrările bazei de date, dar și a testelor automate.
			\item Testele automate pot fi rulate înainte de a se schimba versiunea activă a mediului de testare pe ultima versiune, pentru a se asigura că nu s-a schimbat o parte critică a aplicației.
		\end{itemize}

		Toate aceste motive fac din Heroku mediul meu de testare favorit, ce și-a văzut folosința în cadrul acestui proiect de gestionare a cererilor de despăgubire.
		Mulțumită ușurinței de a schimba variabilele de mediu, am putut să testez de fiecare dată când modificam codul sursă al aplicației schimbările.

		Fiecare modul al aplicației a fost implementată și testată mai întâi în Heroku, iar apoi migrată în producție.

	\subsection{Composer}

		Composer este un utilitar pentru a gestiona referințele externe într-un proiect PHP.
		Permite dezvoltatorului să declare librăriile de care depinde proiectul, pentru a le gestiona (instala sau actualiza) automat.
		Spre diferență de un sistem care gestionează pachetele pentru un sistem de operare bazat pe Linux (precum „Yum” sau „Apt”), acesta instalează pachetele în folderul proiectului numit \verb|„vendor”|, dar permite restrâns instalarea într-un spațiu \verb|„global”|, a unor referințe. \cite{composer}

		Laravel depinde de Composer pentru a gestiona referințele.
		Pentru a porni un nou proiect de Laravel, trebuie să-l instalezi folosind comanda:
		\begin{verbatim}
			composer global require "laravel/installer"
		\end{verbatim}

		Am vorbit despre extensia mediului de dezvoltare PhpStorm prin pachetul „Laravel PhpStorm IDE Helper”.
		Lista completă a pachetelor folosite, în detaliu, în afară de Laravel, a acestui proiect este:

		\begin{itemize}
			\item \verb|barryvdh/laravel-ide-helper| - extensia mediului de dezvoltare, vezi secțiunea în care prezint mediul de dezvoltare PhpStorm.
			\item \verb|doctrine/dbal| - extensie ajutătoare pentru „Laravel PhpStorm IDE Helper”, pentru a afla toate interacțiunile cu modelele Eloquent.
			\item \verb|laravelcollective/html| - extensia „Laravel - Forms \& Html”, întreținută de „Laravel Collective”.
			\item \verb|phpoffice/phpexcel| - extensie ce ajută citirea și scrierea fișierelor în Excel.
			\item \verb|guzzlehttp/guzzle| - extensie ajutătoare pentru AWS S3
			\item \verb|league/flysystem-aws-s3-v3| - principala extensie pentru integrarea Laravel -- AWS S3
			\item \verb|golonka/bbcodeparser| - librărie ce se ocupă de traducerea „BBCode -- HTML” și viceversa.
		\end{itemize}

		După ce se actualizează lista și proiectele folosite în aplicație, Composer poate să ruleze script-uri, prin adăugarea următorului câmp obiect în \verb|composer.json|:
		\begin{Verbatim}
"post-update-cmd": [
	"Illuminate\\Foundation\\ComposerScripts::postUpdate",
	"php artisan ide-helper:generate",
	"php artisan ide-helper:meta",
	"php artisan ide-helper:models -W"
]
	\end{Verbatim}
	Exemplul de mai sus generează codul ajutător pentru mediul de dezvoltare PhpStorm.

	\subsection{Laravel Elixir}

		„npm” este un utilitar de gestionare a librăriilor pentru limbajul de programare JavaScript.
		De asemenea, este utilitarul folosit implicit de mediul de dezvoltare „Node.js”.
		Este compus dintr-un client accesibil prin linie de comandă și o bază de date pe internet formată din pachete publice, numit „npm registry”.
		Registrul este accesat de client, iar pachetele valabile pot fi găsite și accesate prin intermediul website-ului npm.\cite{npm}

		„gulp” este un utilitar găsit pe platforma „npm” ce are ca scop automatizarea elementelor ce necesită mult timp, precum compilarea și mutarea fișierelor dintr-o parte în alta.
		Acesta preferă codul peste configurație, ce asigură un sistem rapid de dezvoltare, fără a scrie în fișiere intermediare pe disc.

		Laravel Elixir a extins interfața sa ușoară de folosit și atunci când vine vorba despre automatizarea cu „gulp”.
		Astfel, acesta știe despre cele mai des folosite procesoare de „CSS” (i.e: „SASS”) și JavaScript (i.e: „Webpack”).
		Folosind sintaxa de a concatena metode una după alta, Laravel Elixir ajută dezvoltatorul a defini propria structură.

		Laravel Elixir mai este folosit și pentru gestionarea foarte ușor versionare fișierelor.
		Asigura de fiecare dată când un client vizitează orice pagină că fișierele de stil și cod sunt actualizate.

		În cadrul aplicației, am transformat simplul fișier dat implicit de Laravel \verb|„gulpfile.js”| într-o platformă elegantă.
		Algoritmul constă în următorii pași:
		\begin{enumerate}
			\item Citesc toate fișierele aflate la calea:
			\begin{verbatim}
				resources/assets/js/pages/*
			\end{verbatim}
			\item
			Construiesc pentru fiecare fișier în parte o variantă a sa minimizată și cu propriul nume (i.e: numele concatenat cu versiunea).
			De asemenea, construiesc pentru principalul stil (\verb|app.scss|) și librăria de cod a aplicației (\verb|app.js|) la fel varianta minimizată cu propriul nume.
			\item
			La final, Laravel Elixir se ocupă de mutarea lor în fișierul public și modificarea referințelor prin fișierul JSON aflat la calea:
			\begin{verbatim}
				/public/build/rev-manifest.json
			\end{verbatim}
		\end{enumerate}

		Folosesc „Webpack”, un utilitar ce combină mai multe fișiere JavaScript într-unul singur.
		Prin intermediul „npm”, am instalat componentele necesare validării timpului, a afișării soluțiilor posibile și a diagramelor prin simplele comenzi:
		\begin{verbatim}
		// pentru sistemul vizual
		require('./bootstrap');

		// pentru validarea timpului
		require('./components/timepicker.min');

		// pentru afișarea soluțiilor posibile
		require('./components/typeahead.min');

		// pentru diagrame
		require('chart.js');

		// extensiile proprii
		require('./extensions');
		\end{verbatim}

	\subsection{LaTeX}

	TeX este un utilitar pentru a redacta documente, conceput de „Donald Knuth”.
	Programul preia un fișier pregătit și-l transformă într-un format ce poate fi tipărit în diferite moduri, cu diferite imprimante.

	LaTeX este un set de macro-uri pentru TeX, conceput pentru a reduce din textul necesar formatării în pagină \cite{latex}.

	Folosind conceptul de la programarea „DRY” („Don't Repeat Yourself” - „Nu Te Repeta”), am împărțit fiecare capitol în fișierul său separat, cu fiecare fișier având la nevoie sub-directoare cu secțiunile importante.

	Pentru a edita documentația licenței, am ales să folosesc „Sublime Text 3”, pentru că este unul dintre cele mai versatile editoare de text în momentul actual.
	Folosind o extensie, m-a  ajutat să pot compila întregul proiect, oriunde m-aș afla în cadrul lui, în formatul PDF („Portable Document Format”), ce poate fi deschis pe orice calculator, cu orice sistem de operare.

	Folosind gândirea de proiect, am setat în „Sublime Text 3” folosirea unui font cu diacritice.
	De asemenea, am activat și descărcat un dicționar pentru Română.
