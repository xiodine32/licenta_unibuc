\section{Mențiuni speciale}

	Fără părinți nu ajungeai cea mai bună variantă a ta.
	Eu n-aș fi putut să obțin tot ceea ce mi-am propus, în timpul alocat, dacă n-ar fi existat aceste utilitare.

	Așa cum pentru Laravel există consola Artisan, pentru gestionarea proiectului am folosit PhpStorm, un mediu de dezvoltare integrat oferit de cei de la JetBrains.

	Pentru a gestiona cunoștințele dobândite despre acest proiect în particular, mai ales ce pași trebuiesc respectați în momentul în care încarc pe server-ul partajat ultima variantă a codului, fără să uit ceva și ulterior codul pentru a face asta automat, am folosit o tablă Kanban.

	Pentru a testa, am folosit Heroku, iar pentru a asigura salvarea eficientă a datelor am folosit AWS („Amazon Web Services”).

	Pentru a include diverse mici proiecte, scrise de alți oameni puși în aceeași situație, am folosit Composer împreună cu npm.

	Iar pentru această lucrare de licență, pe care o citiți în acest moment (vorbind de spargerea celui de-al patrulea zid, nu?), am folosit Sublime Text 3 și \LaTeX.

	\subsection{Google Analytics}

	Pentru a înțelege traficul venind, am ales să folosesc analizatorul gratuit (cu elemente premium) oferit de Google.
	Oferă o sumedenie de detalii despre cum interacționează utilizatorul cu aplicația, de la cât timp petrece un client web pe o anumită pagină, la cele mai des folosite butoane sau cele mai des folosite drumuri (legături între mai multe pagini). \cite{google_analytics}
	De asemenea, poți vedea ce a declanșat o eroare, ceea ce pentru un dezvoltator ajută la identificarea și rezolvarea problemelor din momentul în care apar, nu după ce primești un apel furios de la unul dintre fondatorii companiei.

	Am descoperit, urmărind traficul pentru o săptămână, că punctul sensibil și cel mai des folosit era în momentul în care se căuta o cerere de despăgubire.
	De cele mai multe ori, se știa deja numărul cererii, trebuia doar să se acceseze panoul de căutare și să se introducă numărul în căsuță.

	Astfel am realizat o mică modificare a panoului de navigare principal.
	Se regăsește mulțumită structurii Laravel de organizare a unei aplicații, pe fiecare pagină.
	Modificarea constă în includerea unui câmp completabil rapid.
	După ce s-a completat și utilizatorul a validat codul, se apelează aceeași funcție de căutare în spate, precum accesare panoului de căutarea și introducerea sa în căsuță.

	S-a mai observat durata lungă petrecută de clienți în momentul introducerii datelor cererii de despăgubire.
	Din fericire, Laravel a fost conceput cu un sistem avansat de „uitare” a sesiunii, pentru a preveni abuzurile.
	Dar acesta interfera cu durata lungă în care clienții își completau probabil informațiile și apărea des o eroare a expirării sesiunii.
	Am reușit să previn această problemă apelând constant, prin jQuery, un punct al aplicației ce avea singurul rol de a actualiza sesiunea pentru Laravel, declarând-o activă.
	Utilizator putea în final să stea liniștit și să completeze toate datele, indiferent de cât i-ar fi luat.

	\subsection{Găzduire partajată}

		Găzduirea partajată se folosește pentru a pune un web server conectat la internet valabil pentru mai multe persoane.
		Partajându-se mai multor clienți, se amortizează costul îngrijirii și astfel prețurile sunt reduse.

		Am ales pentru Fandu soluția oferită de xServers\cite{xservers} pentru că nu era doar convenabilă și aveam încredere în ei, pentru că mai găzduisem înainte propria mea pagină, ci pentru certificatele ce asigură calitatea și securitatea.

		Aplicația se mișcă rapid, totul fiind salvat pe servere virtuale cu SSD-uri („Solid State Drive”).

		Un mare avantaj, dar în același timp și dezavantaj este oprirea comenzii de interpretare și executare a codului PHP, \verb|eval|.
		Din păcate, este problematic pentru consola Artisan, pentru că nu poate să aplice în momentul actualizării bazei de date tot codul de migrare.

		Soluția, la care s-a ajuns drept compromis în urma dezbaterilor cu departamentul tehnic, ce migrează codul și baza de date a server-ului, este:
		\begin{enumerate}
			\item
				Se salvează toate datele existente, folosind utilitarul oferit de xServers.
			\item
				Se introduce un fișier nou în directorul aplicației la:
				\begin{verbatim}
					storage/framework/down
				\end{verbatim}
				Se închide astfel accesul la server cât timp se migrează baza de date și conținutul aplicației în sine.
			\item
				Se află codul necesar migrării bazei de date prin comanda:
				\begin{verbatim}
					php artisan migrate --pretend -vvv
				\end{verbatim}
				Ce reiese va fi cod SQL ce va fi copiat cu ușurință în phpMyAdmin-ul de pe server.
			\item
				După ce s-a migrat cu succes baza de date, se copiază următoarele foldere:
					\begin{itemize}
						\item \verb|app| - tot codul important al aplicației.
						\item \verb|bootstrap| - sistemul de încărcare a platformei Laravel.
						\item \verb|config| - setările implicite Laravel, în lipsa valorilor în fișierul mediului curent (\verb|.env|).
						\item \verb|database| - migrările bazei de date
						\item \verb|public| - pentru că folosim „Laravel Elixir”, ce se ocupă de gestiunea variantelor codului JavaScript și CSS, se vor copia și în calea publică \verb|public_html|, pentru că acestea vor fi servite clienților și trebuie să poată să fie accesibile.
						\item \verb|resources| - directivele Blade, codul JavaScript și SASS.
						\item \verb|routes| - rutele ce determină Controller-ul apelat.
						\item \verb|storage| - doar structura, pentru că va conține fișierele temporare traduse din Blade în PHP.
						\item \verb|tests| - se poate sări, pentru că nu afectează rezultatul final.
						\item \verb|vendor| - dacă s-a modificat \verb|composer.json|, adăugând sau modificând sau ștergând pachete.
					\end{itemize}
			\item
				Se va șterge fișierul nou creat:
				\begin{verbatim}
					storage/framework/down
				\end{verbatim}
				și astfel migrația va fi completă.
		\end{enumerate}

		În cazul unor migrări minore, ce nu implică baza de date, se pot suprascrie direct fișierele modificate în cadrul aplicației.
		PhpStorm ajută aici prin opțiunea de a încărca prin FTP (encriptat) fișierele modificate.

		\textbf{\underline{Observație:}} Se copiază directorul \verb|public| de două ori, când se fac modificări la stil sau la codul JavaScript: o dată în cadrul directorului ce conține întregul proiect Laravel și o dată în directorul \verb|public_html|.

		„Laravel Elixir” funcționează bazându-se pe un fișier JSON, ce spune ce versiune de stil și cod să folosească, aflat la:
		\begin{verbatim}
			/public/build/rev-manifest.json
		\end{verbatim}

		Această cale nu este valabilă clienților, pentru că aplicația nu se află în directorul public.
		S-au făcut deci mici modificări asupra fișierului \verb|index.php|, aflat tot în directorul public, pentru a arăta calea corectă.

		Trebuie să se copieze, în concluzie, codul și stilul și în acest director.

	\subsection{PhpStorm IDE}
	\label{sec:phpstorm}
		avantajul de a putea vedea usor ce se intampla cu modelele
		dezvantajul magic method-urilor laravel rezolvat cu proiectul de github.
	\subsection{Heroku}
		rolling release
		testare
		free
	\subsection{Amazon Web Services}
		de ce am ales AWS
		integrarea cu Laravel
		mai intai a fost thumbnail + actual image - imagemagick
		apoi ajungeam in probleme de push commands
		asa ca pana la urma am ales sa mergem pe salvare in chior (fara compression / ceva)
			pentru ca nu depaseam quota-ul de 5GB / luna estimat
	\subsection{npm și Composer}
		\subsubsection{integrarea cu laravel - laravel elixir}
		\subsubsection{ce asigura un up-to-date css / js file cu versioning}
		\subsubsection{agregarea css / scss / js -> webpack (minifier) -> versioning ->}
	\subsection{LaTeX și Sublime Text}
		structura licentei
		legatura cu sublime text
