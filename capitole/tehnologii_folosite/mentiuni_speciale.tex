\section{Mentiuni speciale}
	\subsection{Google Analytics}
	unde se blocheaza / ce partea a interfetei e cea mai des folosita.
	\subsection{Shared hosting}
		avantajul de protectie contra XSS prin disabling exec
		dezavantajul de a nu putea rula migrarile pe serverul de productia
		solutia de urcare pe productie
		solutia de a separa app de public\_html pentru a preveni code exploit-ul.
	\subsection{PhpStorm IDE}
	\label{sec:phpstorm}
		avantajul de a putea vedea usor ce se intampla cu modelele
		dezvantajul magic method-urilor laravel rezolvat cu proiectul de github.
	\subsection{Organizarea cunostiintelor pe un kanban board - Trello}
		construirea unui workflow de a incarca fiecare versiune noua pe serverul de productie
			enable debug / maintenance mode
			aplicare migrari
			incarcare fisiere
			incarcare in public\_html
			disable debug
	\subsection{Heroku}
		rolling release
		testare
		free
	\subsection{Amazon Web Services}
		de ce am ales AWS
		integrarea cu Laravel
		mai intai a fost thumbnail + actual image - imagemagick
		apoi ajungeam in probleme de push commands
		asa ca pana la urma am ales sa mergem pe salvare in chior (fara compression / ceva)
			pentru ca nu depaseam quota-ul de 5GB / luna estimat
	\subsection{npm}
		\subsubsection{integrarea cu laravel - laravel elixir}
		\subsubsection{ce asigura un up-to-date css / js file cu versioning}
		\subsubsection{agregarea css / scss / js -> webpack (minifier) -> versioning ->}
	\subsection{Composer}
	\subsection{LaTeX și Sublime Text}
		structura licentei
		legatura cu sublime text
