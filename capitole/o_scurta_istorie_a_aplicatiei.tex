\chapter{Probleme apărute în timpul dezvoltării aplicației}

	\begin{itemize}
		\item
		S-a introdus verificarea IMEI-ului clienților ce salvau o cerere de despăgubire, datorită cererilor multiple de la aceeași persoană.
		\item
		Inițial, s-au construit rapoarte în format „.csv” (valori delimitate prin virgule).

		Datorită problemelor de compatibilitate între sistemele de operare folosite în cadrul companiei, s-a trecut la exportul de rapoarte în format Excel.

		\item La înregistrarea cererii de despăgubire, fiecărui client îi se asociază un identificator unic, format din 64 caractere.

		Acest „token” era folosit doar pentru a încărca poze, dar s-a adăugat opțiunea și de comunicare client -- utilizator sistem.

		\item
		Din cauza neconcordanțelor coloanelor din tabelul de introdus în sistem, s-a decis a nu se ține cont de numărul coloanelor prestabilit, ci de ce câmpuri se aflau pe prima linie în raport, folosind un algoritm generic dezvoltat.

		\item
		Datorita inconsistentelor rapoartelor de vânzări săptămânale, ce nu conțineau coloane predefinite în structura aplicației, a fost necesară introducerea unui câmpului  „GENERIC”, pentru a evidenția aceste înregistrări.
	\end{itemize}
