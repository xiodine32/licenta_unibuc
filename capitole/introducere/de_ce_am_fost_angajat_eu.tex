\section*{Condițiile preexistente}
	În momentul în care am interacționat prima oară cu Fandu, doreau o redirectare simplă a paginii de internet
	\url{http://www.fandu.uk/claims}\cite{fandu_uk} către \\
	\url{http://wsgp.co.uk/claims/}\cite{wsgp_claims}.

	Proiectul era administrat pe server-ul companiei angajate să scrie codul. Pagina principală încânta utilizatorul cu o avertizare în engleză ce spunea că sistemul ales de gestionare a bazei de date va fi în curând scos din limbajul de programare ales de ei. Nu aveau validări a datelor, nu scoteau rapoartele necesare. Sistemul de gestiune a cererilor de despăgubire era centrat în jurul unor fișiere Excel, ținute de angajați și distribuite între ei.

	Astfel, Fandu și-a dorit să poată să scoată rapoartele dintr-o aplicație. Fără să treacă cineva prin fiecare daună înregistrată. Fără să copieze informațiile dintr-un mail trimis automat. Au dorit să automatizeze o mare parte din munca asiduă a angajaților Fandu. De a putea scoate oricând rapoartele necesare pe orice perioadă de timp, fie ea o lună, un an, sau de la prima cerere de despăgubire.

	Soluția a venit din partea mea. Dorind să mă afirm și să am experiență în domeniu, m-am oferit ca în timpul liber să ajut Fandu. Să scot problema umană din ecuația rapoartelor, să trimit automat mail și să ofer o soluție completă, eficientă, construită de jos în sus pentru fiecare parte a dorințelor lor. Ei au fost sceptici că pot, așa că de menționat ar fi că metoda „rudimentară” de a păstra datele în fișierele Excel a fost activă pe perioada dezvoltării aplicației, pentru a se asigura exactitatea rapoartelor scoase de sistem cu cele „manuale”. Și asta a ajutat enorm. Multe probleme mici au apărut în timpul dezvoltării, ce au fost reparate o dată ce au fost comparate datele.
