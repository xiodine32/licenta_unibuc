\section{Soluțiile software adoptate}

	Pentru a putea reduce costurile aplicației, am optat pentru o metodă tradițională de găzduire partajată, ce oferă PHP și o bază de date MySQL inclusă în pachet.

	După prospectarea librăriilor de dezvoltare software în PHP, proiectul a fost conceput în Laravel\cite{laravel} inspirată din arhitectura sistemului „Ruby on Rails”, care oferă cele mai ridicate standarde de programare.

	Alt avantaj este încercarea minimizării complexității ascunse\cite{laravel_complexity}, ce face codul mult mai ușor de întreținut și permite dezvoltarea mai rapidă, fără a pierde mult timp înțelegând arhitectura sau codul scris anterior.

	Sintaxa ușor de înțeles și ambiguitatea redusă fac din Laravel, în opinia mea, un bun provocator chiar și pentru „ASP.NET MVC”\cite{hotframeworks}, liderul pe piață în momentul actual.
	Ambele platforme oferă soluții de la modele, la gestionarea bazei de date, la rularea migrațiilor, la salvarea stării și chiar și integrarea automată a unui sistem de utilizatori.

	Dar spre diferență de soluția Microsoft, toată platforma necesară construirii și distribuirii aplicației nu este blocată pe sistemul de operare Windows (cu .NET Core situația se schimbă, dar în acest moment nu a fost adoptată la scară largă), ci poate să funcționeze pe orice sistem de operare.
