\section{Despre aplicație}
	Pentru a putea reduce costurile aplicației, am optat pentru o metodă tradițională de găzduire partajată, ce oferă PHP și o bază de date MySQL inclusă în pachet.

	Chiar dacă experiența mea unde lucrez își spune cuvântul în alt limbaj de programare (C\#), am descoperit plictisindu-mă Laravel\cite{laravel}, o platformă modernă de a scrie cod PHP, inspirată din arhitectura sistemului „Ruby on Rails”.

	Înainte să interacționez cu persoanele de la Fandu, aveam deja mai multe proiecte ușoare terminate.
	Știam cât de important este să nu te avânți cu capul înainte, fără a ști un nou cadru în care să programezi, pentru că rezultatul ar fi deplorabil.

	Nu folosisem până acum într-un mediu de producție Laravel, dar eram încrezător că nu mi-ar oferi surprize neplăcute.
	De asemenea, văzusem oportunitatea de a folosi cele mai ridicate standarde de programare pentru PHP, și până la urmă am ales să merg el.

	Astfel, am învățat cum este să gestionezi de unul singur dezvoltarea unei aplicații de la concept la problemele de memorie când nu poți să scoți rapoarte cu peste 30000 de linii, cu baza de date de pe producție.

	Alt avantaj este încercarea minimizării complexității ascunse\cite{laravel_complexity}, ce face codul mult mai ușor de întreținut și permite dezvoltarea mai rapidă, fără a petrece mult timp înțelegând arhitectura sau codul scris anterior.

	Sintaxa ușor de înțeles și ambiguitatea redusă fac din Laravel, în opinia mea, un bun provocator chiar și pentru „ASP.NET MVC”\cite{hotframeworks}, liderul pe piață în momentul actual.
	Ambele platforme oferă soluții de la modele, la gestionarea bazei de date, la rularea migrațiilor, la salvarea stării și chiar și integrarea automată a unui sistem de utilizatori.
	Doar că PHP este mai lent.
