\subsection{Funcții ajutătoare}

	Funcțiile ajutătoare sunt pentru a minimiza timpul dezvoltatorului de lucru.
	Fișierele astfel folositoare sunt:

	\begin{enumerate}
		\item \verb|FormHelper| - Utilitar ce înregistrează o directivă Blade de compunere, ce sunt traduse în simple instrucțiuni de cod PHP.
		Am scris astfel propria directivă de a construi automat codul necesar afișării, folosind HTML și librăria vizuală „Bootstrap”, a unui câmp dintr-un formular, cu nume, tip și validare proprie:
		\begin{verbatim}
			@input(['p_varsta', ['Varsta', true]])
			// unde 'p_varsta' - id-ul câmpului.
			// unde 'Varsta' - denumirea câmpului.
			// unde 'true' - necesitatea completării câmpului.
		\end{verbatim}
		\item \verb|Helpers| -
		Funcțiile ce traduc o variabilă în ceva mai util, precum un șir ce poate fi citit mult mai ușor, dar ce și traduc starea actuală a unei cereri sau decizii în clasa specială CSS.
		\item \verb|TemporaryFiles| - O clasă ce se ocupă de gestiunea fișierelor temporare.

		Este necesară pentru împărțirea în mai multe fișiere a unor cereri mari de date, pentru a nu intra în probleme de memorie.
	\end{enumerate}

	\subsubsection{Request-uri}

	O cerere, presupunând că a fost completată corect pe partea clientului, trebuie să fie verificată și pe partea server-ului.

	În loc să se verifice în fiecare metodă de fiecare dată validitatea cererii, se poate abstractiza și a se face referire la o clasă de tipul Request din Laravel.

	Următoarele clase de tip Request apar, nu sunt goale și sunt folosite în proiect:
	\begin{enumerate}
		\item \verb|ClaimRegisteredRequest| - verificarea existenței unui IMEI înregistrat în ultimele 24 ore.
		\item \verb|StoreClaim| - asigură toate câmpurile unei cereri respectă anumite reguli, declarate folosind sintaxă specială Laravel.
	\end{enumerate}

	\subsubsection{Mediul de dezvoltare}

	Pentru a folosi o aplicație bazată pe Laravel, trebuie să ai configurate anumite servicii, precum tipul de conexiune la o bază de date, ce sistem de mail folosești sau locație aplicației tale pe internet.

	Mulțumită fișierului \verb|„.env”|, nu mai trebuie a te complica cu setarea variabilelor globale sau a variabilelor de mediu, acestea vor fi automat preluate în cazul în care nu există variabila de mediu, din acest fișier.

	O importantă precizare este necesitatea unei chei din acest fișier, \verb|„APP_KEY”|, ce trebuie să existe.

	Această cheie este folosită într-un algoritm de încriptare pentru a face sesiunea clienților și a utilizatorilor cât mai sigură.
