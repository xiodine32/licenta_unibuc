\subsection{Models}

	Baza noastră de date relațională conține multe relații.
	Spre exemplu, cererea de despăgubire poate că este legată la o decizie.
	Eloquent ORM ajută gestionarea și lucrul cu aceste legături foarte ușor și suportă majoritatea legăturilor, printre care:
	\begin{enumerate}
		\item One To One
		\item One To Many
		\item Many To Many
	\end{enumerate}

	Relațiile Eloquent sunt definite ca metode în clasele noastre.
	Precum modelele în sine, acestea sunt folosite precum generatoare de sistem.
	Definind relațiile precum funcții ajută la propagarea metodelor și concatenarea mai multor constrângeri, precum pozele facturii determinate de această metodă aflată în clasa cererii:
\begin{Verbatim}
public function factura_photos()
{
	return $this->hasMany('App\Photo')
		->where('category', Photo::FACTURA);
}
\end{Verbatim}
	În acest caz, pentru că fotografie se poate afla în oricare dintre cele patru categorii, se pune constrângerea ca ea să se afle în categoria facturii.
	Punându-se în legătură mai multe poze la un singur obiect, această relație „One To Many” are nevoie de câmpul \verb|claims_id| să fie prezent în coloana tabelei de poze, conform convenției Laravel. \cite{laravel_relationships}

	Acum că putem să accesăm toate pozele unei cereri, putem defini și relația inversă.
	Schimbând registrul de la poze la decizii, putem accesa având decizia cererea din care provine.
	Pentru asta, se folosește următoarea metodă în clasa deciziei:
\begin{Verbatim}
public function claim()
{
	return $this->belongsTo('App\Claim');
}
\end{Verbatim}
	La fel, pentru că respectă convenției Laravel -- se regăsește câmpul \verb|claims_id| -- se va completa și încărca modelul Eloquent a cererii, atunci când va fi apelată metoda.

	De asemenea, toate metodele folosite sunt „puturoase”.
	Ele nu sunt interpretate și nu rețin rezultatul până când nu au fost chemate cel puțin o dată.

	Presupunând că am dori să avem un număr rotund în loc de cel rotunjit cu virgulă mobilă cu o simplă precizie, am putea să definim un accesor.
	Un accesor cu numele \verb|getFirstNameAttribute| va fi apelat în momentul apelării aflării valorii proprietății \verb|first_name|.
	Cel mai bun exemplu este calcularea valorii de plată nete din modelul „Decision”:
\begin{Verbatim}
public function getNetPayableAttribute()
{
    if ($this->resolution == Decision::REPAIRED) {
        return $this->repair_replace_cost - $this->franchise
            + $this->courier_cost;
    }
    if ($this->resolution == Decision::REPLACED) {
        return $this->payee_cost - $this->franchise
            + $this->courier_cost;
    }

    return "N/A";
}
\end{Verbatim}

	O altă problemă delicată ce ar putea apărea într-un sistem de legare relațională între obiecte și reprezentările bazei de date constă în câmpurile ce pot fi atribuite „la grămadă”.
	O vulnerabilitate „la grămadă” apare atunci când un utilizator mai adaugă să spunem un câmp „Administrator” când se înregistrează și-l setează pe adevărat, ce este trimis apoi în metoda de creare a relației tabel -- obiect, escaladându-și astfel drepturile.

	Pentru a nu fi vulnerabili, putem specifica doar variabilele ce pot fi atribuite „la grămadă”.
	Asta împiedică atribuirea altor câmpuri existente, asupra modelului Eloquent.
	Funcționează incluzând următoarea linie:
	\begin{verbatim}
		protected $fillable = ['nume',  'elemente', 'tabel'];
	\end{verbatim}

	Revenind la proiectul nostru, am mai definit o metodă numită \verb|getReadable|, pentru a afla ce câmpuri sunt obligatorii a fi completate, pentru a construi un nou model.

	Următoarele modele au cod propriu interesant:
	\begin{itemize}
		\item \verb|Decision| - Se folosește de o metodă „Factory” (ce produce obiecte gata pregătite) pentru a construi obiecte din vânzări sau din elemente generice.

		De asemenea se mai calculează prin intermediul metodelor \verb|„remaining”| și \verb|„updateToPay”| sumele rămase, respectiv de plătit.

		\item \verb|Photo| - Se folosește o metodă „Factory” preia extensia și construiește un șir aleator din 32 de caractere, imposibil de ghicit.

		După aceea se atribuie modelului categoria și cererea din care face parte.

		În final, se încarcă pe S3 și se salvează în baza de date.

		De asemenea, este extinsă metoda prin care se șterge modelul, asigurându-se ștergerea sa și de pe AWS S3.

		\item \verb|Sale| - În afară de metoda ce compune un nou generic sale, ce nu mai e folosită, acest model mai oferă și două metode pentru calcularea prețului total al produselor și cantității lor.

		De asemenea, mai oferă încă două metode pentru calcularea prețului total al asigurărilor și a cantităților lor.
	\end{itemize}
