\subsection{Views}

	Gândirea pe care am urmat-o dezvoltând aplicația de gestiune a cereri de despăgubiri, când vine vorba de interfața grafică, a fost bazată pe module.

	De menționat că multe directoare conțin un director numit „includes”.
	Acest director conține directive Blade parțiale ce sunt utilizate pe parcursul afișării, modificării sau interacțiunii cu paginile principale.

	Astfel, interfața e împărțită în mai multe directoare:

	\begin{itemize}
		\item \verb|vendor| - elementele de paginație extrase din context, ce permit păstrarea parametrilor „GET” în momentul folosirii paginației.

		Am ales să fac modificarea implicită de a adăuga doar câmpurile folosite la momentul respectiv pentru că era singurul scenariu în care se puteau pagina rezultatele.

		\item \verb|settings| - elementele vizuale ce afișează setările și interfața de a salva datele aplicației.
		\item \verb|sales| - elementele ce permit vizualizarea, crearea, modificarea și ștergea produselor și asigurărilor, în cazul coruperii datelor.
		\item \verb|reports| - elementele de paginație ce permit construirea rapoartelor.

		Se folosesc de Laravel Elixir pentru a adăuga codul scris în fișierul JavaScript \verb|reports.js| în pagină.

		Astfel se asigură viteza optimă de distribuire a resurselor și funcționalitate cu tipurile de raport.

		Integrarea cu Knockout se face implicit prin declararea includerea din componenta vizuală de șablon a aplicației a fișierului \verb|app.js|.
		\item \verb|newclaim| - elementele de paginație ce sunt vizualizate prima oară de clienții aplicației.

		Directorul \verb|includes| conține împărțit pe componente toate marile arii acoperite de cererea de despăgubire, printre care și un parțial inclus în mail-ul trimis mai departe.
		Acest parțial Blade se referă la actele necesare a fi încărcate în a doua parte a aplicației. (partea de încărcare imagini)
		\item \verb|messages| - elementele de paginație ce sunt incluse cu fiecare tranzacție, ce asigură un canal de comunicare între client și operatorul aplicației.
		\item \verb|mail| - Înăuntrul acestui director se află fișierul \verb|layout.blade.php|, ce conține stilul general valabil tuturor email-urilor trimise.

		De asemenea include pentru fiecare tip de obiect o acțiune desăvârșită ca nume.
		\item \verb|layouts| - conține singurul fișier \verb|app.blade.php|, de care depinde toate celelalte View-uri.

		În cazul în care utilizatorul nu este înregistrat, nu îi se arată bara de navigație în cadrul aplicației de administrare.

		De asemenea, conține referințe la Laravel Elixir, mai ales la stilul și codul precompilat și optimizat de „gulp” (pentru că până la urmă, gulp lucrează cel mai mult).

		În afară de aceste mici modificări, împreună cu Google Analytics și metoda specială de a picura câte o cerere de actualizare a sesiunii la câteva minute, tranformă această pagină a aplicației într-un adevărat utilitar și arată funcționalitatea pe deplină a directivelor Blade.

		\item \verb|import| - elementele de paginație ce sunt vizualizate de cei ce doresc să introducerii date în sistem.

		Directorul \verb|steps| conține împărțit pe componente toate rezultatele posibile introducerii standard de date, dar și celor modificate.

		Directorul \verb|types| conține rezultatele introducerii datelor în sistem, în formatul standard de date.

		\item \verb|errors| - elementele de paginație ce vor fi afișate în caz de erori cu numărul respectiv.

		\item \verb|decisions| - elementele de paginație extrase pentru decizii.

		Conține și o metodă eficientă și rapidă, folosind jQuery , de a sorta informațiile prin apăsarea butoanelor colorate din capul tabelului.
		\item \verb|components| - elementele de paginație extrase din context, ce-mi permite să scriu mai puțin cod pentru a folosi panourile din Bootstrap.
		\item \verb|claims| -  elementele de paginație extrase pentru cereri.

		Conține și o metodă eficientă și rapidă, folosind jQuery , de a sorta informațiile prin apăsarea butoanelor colorate din capul tabelului.

		Directorul \verb|includes| conține componenta de căutare și componenta de modificare a cererii de despăguipre dacă persoana era autorizată (înregistrată și cu o sesiune activă).

		\item \verb|auth| - elementele originale de înregistrare, logare și resetare parolă.
		\item \verb|home.blade.php| - Pagina de introducere când intri în sistem.
	\end{itemize}
.
