\section{Rutina de apelare a programului}

	Pentru a recapitula tehnologiile folosite și pentru a putea intra în detalii despre cum funcționează aplicația per ansamblu, voi detalia cum interacționează un client cu acest sistem de gestiune a cererilor de despăgubire, bazat pe Laravel.
	Cererea pornește de la premiza accesării paginii scrise în regulamentul contractului „Mobile Protect” și parcurge pașii unui client web (ceea ce nu se observă când se încarcă pagina).
	\begin{enumerate}
		\item
			Clientul navighează spre \url{www.fandu.uk/claims}.
		\item
			El este redirectat spre \url{https://fandu.info}, prin intermediul „header”-ului, pagina livrată fiind goală.
		\item
			Folosind un sistem distribuit de nume de domeniu, presupunând că domeniul nu a fost descoperit până acum de client, se face cererea către CloudFlare.
			Presupunem că nu este un atac malițios, altfel ar fi fost oprit în acest punct și ar fi fost posibil obligat să rezolve o serie de întrebări pentru a continua.
			Depinde de severitatea atacului.
		\item
			Aplicația Laravel se încarcă. Își identifică astfel mediul de lucru și variabilele definite.
		\item
			Se conectează la baza de date MySQL și încearcă să identifice sesiunea clientului, în funcție de „cookie”-ul denumit „laravel\_session”.
			Pentru că presupunem că este un client nou, nu va găsi o sesiune deja existentă.
			Nu este un client înregistrat.
		\item
			Laravel observă rutele predefinite și încearcă să găsească o rută ce se potrivește căii, reținând contextul.
		\item
			În acest moment, Laravel deține numele Controller-ului, numele metodei ce va fi apelate și toate denumirile scurte ale claselor intermediare ce trebuiesc validate pentru a ajunge cererea la metoda clasei, aceste clase mijlocii fiind denumite „middleware”.

			Un exemplu bun al unei clase middleware este „auth”, ce asigură un utilizator înregistrat, cu o sesiune activă.
		\item
			Se construiește Controller-ul folosind calea determinată la pasul anterior și se actualizează (în cazul în care apare cod de adăugare unei clase „middleware” în interiorul metodei ce construiește obiectul) lista de „middleware”.
		\item
			Se construiește și se validează fiecare clasă „middleware”, iar în cazul unui eșec, nu se va continua mai departe.
		\item
			Se apelează metoda clasei asigurându-se prin intermediul unui obiect ce rezolvă legăturile dintre obiectele dorite și cele existente trimiși ca parametrii metodei Controller-ului.
			Se va primi un răspuns.
		\item
			În cazul în care răspunsul nu este o directivă Blade, se va afișa sau va redirecționa clientul web.
			Altfel, se va compila directiva Blade dacă nu există fișierul deja compilat, sau dacă timpul ultimei modificări directivei e mai recent decât timpul ultimului rezultat al compilării.
	\end{enumerate}
