\section{Structura Model-View-Controller}

	Revenind la definiția arhitecturii, „Model-View-Controller” se bazează pe separarea puterii celor trei componente definitorii pentru o aplicație: Modelul, View-ul, respectiv Controller-ul.\cite{poo_sa}

	Modelul este componenta centrală a șablonului.
	Ea se ocupă de datele aplicației, iar mulțumită Laravel Eloquent, face legătura dintre baza de date și aplicație.

	Modelele sunt gestionate de Controller, ce acceptă datele de la utilizator și le transformă în comenzi pentru model și/sau View.

	View-ul este orice metodă de vizualizare a informației unui Model -  în cazul nostru directivele Blade.

	Laravel construiește astfel aplicații web bazate pe arhitectura „Model-View-Controller”.
	Deci și această aplicație respectă arhitectura.

	În cadrul aplicației următoarele căi sunt notabile:
	\begin{itemize}
		\item Model - \begin{verb} /app/ \end{verb}
		\item View - \begin{verb} /resources/views/ \end{verb}
		\item Controller - \begin{verb} /app/Http/Controllers/ \end{verb}
	\end{itemize}

	\import{./capitole/structura_aplicatiei/mvc/}{controllers}
	\import{./capitole/structura_aplicatiei/mvc/}{models}
	\import{./capitole/structura_aplicatiei/mvc/}{views}
	\import{./capitole/structura_aplicatiei/mvc/}{helpers}
