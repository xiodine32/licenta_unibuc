\section{Teste automate}
	În lunile care vin, înainte de mă ocupa de sistemul modular de rapoarte, o să doresc să petrec o bună perioadă scriind teste automate.

	Testele automate, mai ales „testele unitare”, sunt o metodă de verificare a modulelor individuale, asocierilor datelor, procedurilor de modificare a datelor. \cite{testare_automata}

	Acestea pornesc de la premiza că un modul detașat de modificările externe, odată testat, se va comporta corect și atunci când va fi folosit în conjuncție cu alte module, ce la rândul lor vor fi testate. Astfel, de la cele mai mici module, se asigură comportamentul corect al aplicației. De la nivelul microscopic de funcție, până la nivelul macroscopic de funcționalitate.

	Doresc să extind aplicația să se folosească de codul gata scris de cadrul Laravel și de a obliga aplicația găzduită pe mediul de testare să se asigure că fiecare nouă linie de cod nu strică funcționalitatea deja existentă. Acest principiu se mai numește și „integrare continuă”, propusă prima oară de  Grady Booch în 1991 \cite{integrare_continua} și folosită de majoritatea dezvoltatorilor moderni.
